%!TEX root = ../dokumentation.tex
\chapter{Einführung}
Wir leben in einer Zeit in der der Einsatz von Technik im Alltag aus unserem Leben nicht mehr weg zu denken ist. 
Technologien wie selbstfahrende Autos, die vor wenigen Jahren noch als Science Fiction klassifiziert wurden, sind nun realisierbar. 
In Fabriken werden selbst komplexere Tätigkeiten von Robotern übernommen und zu Hause gehört das Staubsaugen der Vergangenheit an, 
weil dies eine programmierte Maschine zuverlässig übernehmen kann. \\
In den verschiedensten Situationen erlebt unser Leben eine Bereicherung und Erleichterung durch neue Technologien. So auch im Gesundheitswesen. 
In Asien werden beispielsweise Roboter zur Altenpflege eingesetzt weil es dort wie auch in Deutschland an Pflegekräften mangelt. 
Hier in Deutschland ist der Einsatz moderner Technik im Gesundheitswesen allerdings noch nicht üblich, gäbe es doch so viele denkbare Fälle in denen kranken Menschen durch mehr Forschung auf diesem Gebiet sehr geholfen werden könnte.\\
Das Projekt hinter dieser Studienarbeit soll einen Teil dazu beitragen, das Leben kranker, genauer hötgeschädigter, Menschen zu erleichtern.\par
Hörgeschädigte Menschen haben ein kleines defiziet, welches allerdings dazu führt, dass Ihnen das Teilnehmen am sozialleben sehr erschwert wird: 
Ihr Gehör funktioniert nicht. Angefangen beim Schulbesuch, wo ein Gebärdesprachendolmetscher von nöten ist, damit sie dem Untericht folgen können über Dialoge XXXXXXX \par
Weil im Moment an einer Brille entwickelt wird, die mit einer Kamera bestückt ist, welche das gesehene Bild aufzeichnet, live verarbeitet und dem Träger erfasste Texte vorliest und ihn über vor ihm stehende Personen informiert, 
war die Idee nicht weit, diese Konvertierung in die entgegengesetzte Richtung vorzunehmen: 
Der von einem Mobiltelefon aufgenommene Ton wird auf das Vorkommen von Sprachbausteinen untersucht und die erkannte Sprache in Text konvertiert. 
Der Text wird auf einer mit dem Mobiltelefon verbundene AR-Brille ausgegeben.\\
So kann ein hörgeschädigter jedem Vortrag, jeder Unterichtsstunde problemlos folgen. 
Auch das kommunizieren mit Menschen, die der Gebärdensprache nicht mächtig sind, wird vereinfacht, da diese einfach in das Microphone sprechen müssen. 
Durch die Integration des Konvertierers in ein Mobiltelefon, würde es sogar möglich, dass ein Hörgeschädigter nun Telefonieren könnte, da die Stimme des Anrufers direkt verarbeitet werden könnte und dem Anwender live als Text auf die AR-Brillengläser projekziert werden können.

