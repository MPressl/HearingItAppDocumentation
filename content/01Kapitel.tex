%!TEX root = ../dokumentation.tex
\chapter{Einführung}
Wir leben in einer Zeit, in der der Einsatz von Technik im Alltag nicht mehr weg zu denken ist. Die Menschheit ist so stark vom technischen Fortschritt und den daraus resultierenden Entwicklungen geprägt wie nie zuvor.  
Technologien wie Autos, zu deren Betrieb eine körperliche Interaktion nicht mehr von Nöten ist, wurden vor wenigen Jahren noch als Science Fiction klassifiziert und sind trotz allem heute realisierbar. Auch In Fabriken werden selbst komplexere, erfolgsentscheidende Tätigkeiten von Robotern umgesetzt und zu Hause gehört das eigenständige Staubsaugen der Vergangenheit an, weil dies eine programmierte Maschine zuverlässig übernehmen kann. \\Neue Technologien wirken sich in vielen Alltagssituationen bereichernd und erleichternd auf unser Leben aus. Ein hiervon stark betroffener Wirtschaftssektor ist das wissens- und technologiegetriebene Gesundheitswesen. Der medizintechnische Fortschritt, der sich in den letzten Jahrzehnten ereignet hat, ermöglicht weitaus zielführendere Diagnose- und Therapiemöglichkeiten als sie früher vorstellbar waren. So lassen sich beispielsweise in der diagnostischen Fachabteilung Radiologie mit dem Einsatz neuer Technologien, wie Computer- und Kernspintomographien, aussagekräftigere Befunde erzielen als sie ein Pathologe früher durch einen händischen Eingriff hätte anfertigen können \cite[S.3]{buck_radiologie_2013}.\\Die Statistiken des statistischen Bundesamtes zur durchschnittlichen Lebenserwartung der Bevölkerung Deutschlands zeigen einen stetigen Aufschwung im Verlauf der vergangenen Jahrzehnte und lassen daher auf einen erheblichen Anstieg der allgemeinen Patientenversorgungsqualität im Gesundheitswesen aufgrund der Entwicklung IT-basierter Medizintechnologien schließen.\\ 
Diese Tatsache gilt als ausschlaggebender Beweggrund dafür, die Studienarbeit im Rahmen eines medizininformatischen Entwicklungsprojektes zu absolvieren. Der Fokus ist hierbei auf Menschen gerichtet, die von Hörschädigungen betroffen sind und tagtäglich mit ihren schwerwiegenden Folgen konfrontiert werden. Soziale Integrationsschwierigkeiten und psychische Beeinträchtigungen bilden nur das Grundgerüst der gängigen Symptomatik. Das Ziel des Projekts besteht darin, den oben genannten Belastungen durch Hörstörungen mithilfe eines IT-Produktes entgegenzuwirken.\\Als Entwicklungsgrundlage dient eine Technologie mit dem Namen Augmented Reality (zu Deutsch: Erweiterte Realität). Darunter lässt sich eine Variation der Ursprungstechnologie Virtual Reality verstehen, mit deren Hilfe Brillen entwickelt werden können, die dem User optisch das Gefühl vermitteln sich in einer virtuellen Welt zu befinden. Die reale Welt um ihn herum wird dabei vollständig ausgeblendet und ist für ihn somit nicht übersehbar.\\Augmented Reality (kurz: AR) sah dies als Schwachstelle und versuchte dem entgegenzuwirken. 
Durch AR lassen sich digitale Daten oder computergenerierte Informationen auf Brillen visualisieren und in die reale Welt integrieren. Gegenüber einer VR-Brille ist die reale Welt Teil der AR-Technologie und somit für den User sichtbar. AR erlaubt die Erfassung und Verarbeitung von Daten, beispielsweise die Aufzeichnung und Analyse eines gesehenen Bildes, und ermöglicht die Projizierung der Ergebnisse auf die Brillengläser \cite{kipper_augmented_2012}. Gleichermaßen lässt sich diese Konvertierung auch in die entgegengesetzte Richtung vornehmen, wie es auch im hier behandelten Studienprojekt verwendet wird:\\ 
Der von einem Mobiltelefon aufgenommene Ton wird auf das Vorkommen von Sprachbausteinen untersucht und die erkannte Sprache in Text konvertiert. Der Text wird auf einer AR-Brille ausgegeben, die in Verbindung mit dem Mobiltelefon steht. So kann ein Mensch auch mit Hörschädigungen einem Vortrag oder einer Unterrichtsstunde akustisch und visuell problemlos folgen. Auch das kommunizieren mit Menschen, die der Gebärdensprache nicht mächtig sind, wird vereinfacht, da diese einfach in das Mikrofon des Mobiltelefons sprechen müssen. Durch die Integration des Konvertierers in ein Mobiltelefon, wäre es sogar möglich beim Telefonat die Stimme des Anrufers direkt zu verarbeiten und dem Hörgeschädigten live als Text auf die AR-Brillengläser zu projizieren.

